\documentclass[a4paper,12pt]{article}
\usepackage[UTF8]{ctex}     % 支持中文
\usepackage{geometry}       % 页面设置
\usepackage{titlesec}       % 标题设置
\usepackage{color}          % 颜色支持
\usepackage{listings}       % 代码块支持
\usepackage{graphicx}       % 图片支持
\usepackage{float}          % 图片浮动位置
\usepackage{hyperref}       % 超链接
\usepackage{fancyhdr}       % 页眉页脚
\usepackage{booktabs}       % 用于更漂亮的表格线
\usepackage{amsmath}        % 用于数学公式
\usepackage{array}          % 用于表格列宽调整
\usepackage{subfigure}      % 用于子图排列

% 页面边距设置
\geometry{left=2.5cm,right=2.5cm,top=2.5cm,bottom=2.5cm}

% 代码块样式设置
\definecolor{codegreen}{rgb}{0,0.6,0}
\definecolor{codegray}{rgb}{0.5,0.5,0.5}
\definecolor{codepurple}{rgb}{0.58,0,0.82}
\definecolor{backcolour}{rgb}{0.95,0.95,0.92}

\lstdefinestyle{mystyle}{
    backgroundcolor=\color{backcolour},   
    commentstyle=\color{codegreen},
    keywordstyle=\color{magenta},
    numberstyle=\tiny\color{codegray},
    stringstyle=\color{codepurple},
    basicstyle=\ttfamily\footnotesize,
    breakatwhitespace=false,         
    breaklines=true,                 
    captionpos=b,                    
    keepspaces=true,                 
    numbers=left,                    
    numbersep=5pt,                  
    showspaces=false,                
    showstringspaces=false,
    showtabs=false,                  
    tabsize=4
}
\lstset{style=mystyle}

% 页眉页脚设置
% \pagestyle{fancy}
% \fancyhf{}
% \lhead{计算机系统结构实验}
% \rhead{ArchLab Part C 实验报告}
\cfoot{\thepage}

% 标题信息
\title{\textbf{计算机组成原理 ArchLab 实验报告}}
\author{\textbf{姓名}:李孙木铎 \quad \textbf{学号}:2023010304 \quad \textbf{班级}:软件31}

\date{\today}

\begin{document}

\maketitle

\setcounter{page}{1}
\section{实验简介}
本实验旨在深入理解流水线 Y86-64 处理器的设计与实现,并通过优化基准程序和处理器架构来最大化性能。
实验分为三个部分:Part A 熟悉 Y86-64 汇编编程;Part B 扩展 SEQ 模拟器以支持新指令 \texttt{iaddq};
Part C 则是实验的核心,通过修改 \texttt{ncopy.ys} 和 \texttt{pipe-full.hcl} 来优化流水线性能(CPE)。

\section{Part A: Y86-64 汇编指令编程}

\subsection{任务描述}
Part A 要求实现三个简单的 Y86-64 程序,分别是:
\begin{itemize}
    \item \texttt{sum.ys}: 迭代计算链表元素之和。
    \item \texttt{rsum.ys}: 递归计算链表元素之和。
    \item \texttt{copy.ys}: 将源内存块复制到目标内存块,并计算 checksum(异或和)。
\end{itemize}

\subsection{实现思路}
\begin{itemize}
    \item 由于 Y86-64 指令集是一般 x86-64 的简化版本,也即 x86-64 的子集,因此可以参考 C 语言的实现逻辑,将其逐步翻译为汇编指令;
    \item 可以调用gcc的汇编工具,将原本的 C 代码编译为 x86-64 汇编代码,分析关键逻辑部分的汇编实现;
    \item 之后,结合 Y86-64 指令集的限制,进行相应的指令替换和逻辑调整,最终实现所需功能。
\end{itemize}

\subsection{实现细节}

\subsubsection{\texttt{sum.ys} (迭代求和)}
该程序实现了迭代版本的链表求和。
\begin{itemize}
    \item \textbf{参数传递}:链表头指针 \texttt{ls} 通过 \texttt{\%rdi} 传递。
    \item \textbf{循环逻辑}:使用 \texttt{andq \%rdi, \%rdi} 检查当前节点是否为空。若非空,则读取 \texttt{(\%rdi)} 到临时寄存器并累加到 \texttt{\%rax},然后更新 \texttt{\%rdi} 为 \texttt{8(\%rdi)}(即 \texttt{next} 指针),直到链表结束。
    \item \textbf{栈空间与数据}:参考实验文档,将待测数据加载在数据段中,并按照规范合理调用主函数,分配栈空间。
\end{itemize}
\textbf{实验结果}:


\subsubsection{\texttt{rsum.ys} (递归求和)}
该程序实现了递归版本的链表求和。
\begin{itemize}
    \item \textbf{保存现场}:由于是递归调用,必须将 \texttt{val}(当前节点值)保存在被调用者保存寄存器(如 \texttt{\%rbx})中,并在调用前将 \texttt{\%rbx} 压栈保存。
    \item \textbf{递归逻辑}:
        \begin{enumerate}
            \item 若 \texttt{ls} 为空,直接跳转至返回 0 。
            \item 否则,取出当前值存入 \texttt{\%rbx},加载 \texttt{next} 指针到 \texttt{\%rdi},调用 \texttt{call rsum\_list}。
            \item 将两个值相加后存入返回值寄存器 \texttt{\%rax},并恢复现场(弹出栈中的 \texttt{\%rbx})。
        \end{enumerate}
\end{itemize}

\subsubsection{\texttt{copy.ys} (块复制与 Checksum)}
该程序实现了内存块复制及异或校验和计算。
\begin{itemize}
    \item \textbf{参数}:\texttt{src} (\texttt{\%rdi}), \texttt{dest} (\texttt{\%rsi}), \texttt{len} (\texttt{\%rdx})。
    \item \textbf{常量加载}:由于 Y86-64 不支持直接的立即数加载,因此先使用 \texttt{irmovq} 将常量加载到闲置寄存器(如 \texttt{\%r8})中备用。
    \item \textbf{逻辑}:使用循环结构,每次从 \texttt{src} 读取一个字,写入 \texttt{dest},并与 \texttt{\%rax} 进行异或运算。
    之后,\texttt{src} 和 \texttt{dest} 指针增加 8,\texttt{len} 减 1。
\end{itemize}

\subsection{实验结果}
三个程序经\texttt{yas}编译器和\texttt{yis}模拟器后,
输出结果如\hyperref[fig:part_a_results]{图~\ref{fig:part_a_results}}所示:

\begin{figure}[H]
    \centering
    \subfigure[sum.ys 运行结果]{
        \includegraphics[width=0.5\textwidth]{images/sum.png}
    }
    \subfigure[rsum.ys 运行结果]{
        \includegraphics[width=0.5\textwidth]{images/rsum.png}
    }
    \subfigure[copy.ys 运行结果]{
        \includegraphics[width=0.5\textwidth]{images/copy.png}
    }
    \caption{Part A 三个程序的运行结果}
    \label{fig:part_a_results}
\end{figure}
可见,最终 \texttt{\%rax} 中的值为 \texttt{0x0000000000000cba},即链表元素之和(或内存块校验和);
且对内存块复制的结果,目的内存区的数据与源内存区一致,均符合预期。

\section{Part B: 在 SEQ 处理器中实现 iaddq 指令}

\subsection{任务描述}
Part B 要求在顺序处理器(SEQ)的硬件描述语言 \texttt{seq-full.hcl} 中实现 \texttt{iaddq V, rB} 指令。
该指令的功能是 \texttt{R[rB] $\leftarrow$ R[rB] + V}。由于 Y86-64 指令集中没有直接支持立即数加法的指令,
因此实现该指令后,可以减少 \texttt{irmovq} 指令的使用频率,从而提升程序性能。

\subsection{iaddq 指令逻辑分析}
参考 CSAPP 练习题 4.3,\texttt{iaddq} 指令总长度为 10 字节,编码格式如下:
\begin{itemize}
    \item \textbf{指令编码}:\texttt{C 0}(1字节);
    \item \textbf{寄存器编码}:\texttt{F rB}(1字节,\texttt{rA} 字段无效,填 \texttt{F});
    \item \textbf{立即数}:\texttt{V}(8字节,64位有符号整数).
\end{itemize}

各阶段的计算逻辑如\hyperref[tab:iaddq_seq_logic]{表~\ref{tab:iaddq_seq_logic}}所示:
\begin{table}[H]
    \centering
    \caption{SEQ处理器中iaddq指令在各个阶段的计算逻辑}
    \label{tab:iaddq_seq_logic}
    \renewcommand{\arraystretch}{1.5} % 增加行高,使公式不拥挤
    \begin{tabular}{@{}ll@{}}
        \toprule
        \textbf{Stage} & \textbf{\texttt{iaddq V, rB}} \\ 
        \midrule
        \textbf{Fetch} & $\text{icode:ifun} \leftarrow M_1[\text{PC}]$ \\
                       & $\text{rA:rB} \leftarrow M_1[\text{PC} + 1]$ \\
                       & $\text{valC} \leftarrow M_8[\text{PC} + 2]$ \\
                       & $\text{valP} \leftarrow \text{PC} + 10$ \\ 
        \midrule
        \textbf{Decode} & $\text{valB} \leftarrow R[\text{rB}]$ \\ 
        \midrule
        \textbf{Execute} & $\text{valE} \leftarrow \text{valB} + \text{valC}$ \\
                         & $\text{Set CC}$ \\ 
        \midrule
        \textbf{Memory} &  \\ 
        \midrule
        \textbf{Write Back} & $R[\text{rB}] \leftarrow \text{valE}$ \\ 
        \midrule
        \textbf{PC Update} & $\text{PC} \leftarrow \text{valP}$ \\ 
        \bottomrule
    \end{tabular}
\end{table}

\subsection{HCL 实现细节}
在 \texttt{seq-full.hcl} 中,我添加了 \texttt{IIADDQ} 到相应的控制信号集合中:
\begin{lstlisting}[language=C]
# Fetch Stage
bool instr_valid = icode in { INOP, IHALT, ..., IIADDQ };
bool need_regids = icode in { IRRMOVQ, ..., IIADDQ };
bool need_valC   = icode in { IIRMOVQ, IRMMOVQ, IMRMOVQ, IIADDQ };

# Decode Stage
srcB = [
    icode in { IOPQ, IRMMOVQ, IMRMOVQ, IIADDQ } : rB;
    ...
];
dstE = [
    icode in { IIRMOVQ, IOPQ, IIADDQ } : rB;
    ...
];

# Execute Stage
aluA = [
    icode in { IIADDQ, IIRMOVQ, IMRMOVQ, IRMMOVQ } : valC;
    ...
];
aluB = [
    icode in { IIRMOWQ, IMRMOVQ, IOPQ, IIADDQ } : valB;
    ...
];
set_cc = icode in { IOPQ, IIADDQ };
\end{lstlisting}

\subsection{测试验证}
使用工具测试\texttt{iaddq}指令的正确性,并验证整个处理器的功能完整性。测试结果如\hyperref[fig:part_b_test]{图~\ref{fig:part_b_test}}所示:
\begin{figure}[h]
    \centering
    \subfigure[验证标准指令集完整性]{
        \includegraphics[width=0.6\textwidth]{images/testssim.png}
    }
    \subfigure[验证 iaddq 指令正确性]{
        \includegraphics[width=0.6\textwidth]{images/testiaddq.png}
    }
    \caption{Part B iaddq 指令测试结果}
    \label{fig:part_b_test}
\end{figure}

\section{Part C: 流水线处理器设计与代码性能优化}

\subsection{任务描述}
Part C 的目标首先是修改 \texttt{pipe-full.hcl},使流水线处理器支持 \texttt{iaddq} 指令;
其次是优化 \texttt{ncopy.ys} ,使其在复制内存块并统计正数个数时的性能(CPE, Cycles Per Element)最优。目标 CPE 为 7.50 以下。

\subsection{优化策略分析}
为了达到极致的性能,我主要采用了以下几种优化手段:

\subsubsection{循环展开 (Loop Unrolling)}
循环展开可以在一个循环中处理多个元素,提高寄存器利用率,进而提高指令级并行性,减少分支跳转带来的控制冒险。
经分析可知,对15个寄存器,一共有11个寄存器可以用于并行进行内存加载和存储操作,因此最多可以展开到11路。
但考虑后续的余数处理,选择8路或9路展开更为合适。

% \subsubsection{指令调度与“三明治”写法 (Instruction Scheduling)}
% 在PIPE中,加载-使用(Load-Use)冒险和控制冒险(Control Hazard)会向流水线引入气泡,降低性能。
% \begin{itemize}
%     \item \textbf{问题}:按处理逻辑,常规写法是 \texttt{mrmovq} $\rightarrow$ \texttt{andq} $\rightarrow$ \texttt{jle}。
%     这会导致 \texttt{mrmovq} 和 \texttt{andq} 之间有一个由加载-使用冒险引入的气泡。
%     \item \textbf{优化(三明治写法)}:调整指令顺序,将存储指令 \texttt{rmmovq} 插入到条件判断和跳转之间。
%     \begin{lstlisting}
% mrmovq (%rdi), %r8      # Load
% andq %r8, %r8           # Check (Update CC)
% rmmovq %r8, (%rsi)      # Store (Useful work acting as NOP for CC stability)
% jle Npos                # Jump\end{lstlisting}
%     这种 \texttt{Load} $\rightarrow$ \texttt{Check} $\rightarrow$ \texttt{Store} $\rightarrow$ \texttt{Jump} 的顺序,利用 \texttt{rmmovq} 不改变条件码的特性,完美掩盖了流水线气泡。
% \end{itemize}

\subsubsection{三叉树余数处理 (Ternary Tree Remainder Handling)}
循环展开后,剩余元素(0-8个)的处理成为短数组性能的瓶颈。简单的线性查找效率太低。
由于条件跳转指令可以基于比较结果跳转到三组不同的标签,因此可以设计一棵三叉搜索树来快速定位剩余长度的处理入口。
(选择9路展开,对三叉树结构更为自然;如果选择8路展开,则需要构建深度为3的二叉树,复杂度更高。)
基于 \texttt{iaddq} 判断余数大小,设计三叉搜索树:
\begin{itemize}
    \item 首先判断 \texttt{len < 3},若是则跳转到处理 0-2 个元素的标签;
    再判断 \texttt{len < 6},若是则跳转到处理 3-5 个元素的标签;
    否则跳转到处理 6-8 个元素的标签;
    \item 对每个标签内,继续使用 \texttt{iaddq} 判断具体的剩余数量,并跳转到对应的单个处理块。
    \item 对处理部分,采用倒序阶梯式处理,降低跳转指令的数量。
\end{itemize}

\subsubsection{预加载技术 (Pre-loading)}
% 在余数处理的跳转树中,在\textbf{跳转发生之前}就提前执行 \texttt{mrmovq} 将下一层可能用到的数据加载到对应的寄存器中。
\begin{itemize}
    \item \textbf{问题}:由于加载-使用冒险,在加载指令\texttt{mrmovq} 后,紧接着使用该数据的指令(如 \texttt{andq})会引入一个气泡。
    \item \textbf{解决}:可以使用一个不影响条件码的指令(如 \texttt{je})来填补这个气泡,从而保持流水线的连续性。
    在余数处理的过程中,将所有的\texttt{mrmovq}指令都放在跳转指令之前,也即进行数据的预加载,就可以有效减少气泡的产生。
    \item \textbf{效果}:只应用三叉树和循环展开,CPE 可降至约 7.65;
    加入预加载后,CPE 进一步降至约 7.49,达到满分要求。
\end{itemize}


\subsubsection{瀑布流处理 (Waterfall Structure)}
对于剩余元素的具体处理,构建了一个紧凑的阶梯式处理结构:
每个处理块(RK)在处理完第 K 个元素后,会加载第 K-1 个元素到寄存器,
之后代码会自然执行到下一个处理块(Handle K-1)。这个设计减少了跳转指令的使用,同时预防了加载-使用冒险。

\subsection{代码实现概览}
以下是 \texttt{ncopy.ys} 的核心代码片段:

\begin{lstlisting}
Loop:
    # 9-way Unrolling with "Sandwich" Optimization
    mrmovq (%rdi), %r8 
    # ... (Load all 9 values) ...
    mrmovq 64(%rdi), %rbx

    rmmovq %r8, (%rsi)   # Store Val 0
    # ... (Store all 9 values) ...
    rmmovq %rbx, 72(%rsi) # Store Val

Npos0:
	andq %r8, %r8		# val0 <= 0?
	jle Npos1		    # if so, goto Npos0:
	iaddq $1, %rax 		# count++
Npos1:
    # ... (Repeat for all 9 values) ...
Npos9:
    # Loop update
	iaddq $72, %rdi			# src += 8 * 10
	iaddq $72, %rsi			# dst += 8 * 10
	iaddq $-9, %rdx		    # len -= 10
	jge Loop		        # if so, goto Loop:

# Remainder Handling with Pre-loading
Remainder:
    iaddq $6, %rdx          # <0 (-9..-7 -> 剩0..2)
    jl Rem_0_2

Rem_3_8: # rdx now ranges from 0 to 5
	iaddq $-3, %rdx         # <0 (0..2 -> 剩3..5)
	jl Rem_3_5

Rem_6_8: # rdx now ranges from 0 to 2
	iaddq $-1, %rdx         # rdx now ranges from -1 to 1
	mrmovq 40(%rdi), %r13
	jl R6 					# <0 (剩6)
	mrmovq 48(%rdi), %r14
	je R7					# =0 (剩7)
	mrmovq 56(%rdi), %rbx
	jg R8					# >0 (剩8)
# ... (Similar handling for Rem_3_5 and Rem_0_2) ...

R8: # handle 8th remaining element
	andq %rbx, %rbx
	rmmovq %rbx, 56(%rsi)
	mrmovq 48(%rdi), %r14
	jle R7
	iaddq $1, %rax
# ... (Similar handling for R7, R6, R5, R4, R3, R2, R1) ...
\end{lstlisting}

\subsection{pipe-full.hcl 的修改}
为了支持 \texttt{ncopy.ys} 中的优化,需在 \texttt{pipe-full.hcl} 中实现了 \texttt{iaddq} 指令。
各流水线阶段的实现逻辑与关键 HCL 信号如\hyperref[tab:pipe_iaddq_logic]{表~\ref{tab:pipe_iaddq_logic}}所示:

\begin{table}[h]
    \centering
    \caption{流水线模式下 iaddq 指令的各阶段实现逻辑与 HCL 信号}
    \label{tab:pipe_iaddq_logic}
    \renewcommand{\arraystretch}{1.5}
    \begin{tabular}{@{}l p{6cm} l@{}}
        \toprule
        \textbf{Stage} & \textbf{Operation Logic} & \textbf{Key HCL Signals} \\ 
        \midrule
        \textbf{Fetch} & $\text{icode:ifun} \leftarrow M_1[\text{PC}]$ & \texttt{instr\_valid = 1} \\
                       & $\text{rA:rB} \leftarrow M_1[\text{PC} + 1]$ & \texttt{need\_regids = 1} \\
                       & $\text{valC} \leftarrow M_8[\text{PC} + 2]$ & \texttt{need\_valC = 1} \\
                       & $\text{valP} \leftarrow \text{PC} + 10$ & \\ 
        \midrule
        \textbf{Decode} & $\text{valB} \leftarrow R[\text{rB}]$ & \texttt{d\_srcB = D\_rB} \\
                        & $\text{dstE} \leftarrow \text{rB}$ & \texttt{d\_dstE = D\_rB} \\ 
        \midrule
        \textbf{Execute} & $\text{valE} \leftarrow \text{valB} + \text{valC}$ & \texttt{aluA = E\_valC} \\
                         & $\text{Set CC}$ & \texttt{aluB = E\_valB} \\
                         & & \texttt{alufun = ALUADD} \\
                         & & \texttt{set\_cc = 1} \\ 
        \midrule
        \textbf{Memory} & \textit{No Operation} & \texttt{mem\_read = 0} \\
                        & & \texttt{mem\_write = 0} \\ 
        \midrule
        \textbf{Write Back} & $R[\text{dstE}] \leftarrow \text{valE}$ & (\textit{Handled by pipeline}) \\ 
        \bottomrule
    \end{tabular}
\end{table}

\subsection{实验结果}
编译构建 PIPE 处理器,经过正确性测试和性能测试,最终结果如\hyperref[fig:part_c_result]{图~\ref{fig:part_c_result}}所示:
\begin{figure}[h]
    \centering
    \subfigure[正确性测试结果]{
        \includegraphics[width=0.8\textwidth]{images/correctness.png}
    }
    \subfigure[性能测试结果]{
        \includegraphics[width=0.8\textwidth]{images/mark.png}
    }
    \label{fig:part_c_result}
    \caption{Part C 正确性与性能测试结果}
\end{figure}
且程序编译后字节长度为967字节,符合实验要求。

\section{总结}
\begin{itemize}
    \item 通过 Y86-64 汇编编程,更深入地理解了汇编语言与底层计算机体系结构的关系,同时熟悉了Y86-64指令集的特点与限制。
    \item 通过在 SEQ 处理器中实现 \texttt{iaddq} 指令,理解了处理器实现指令集的基本流程与关键技术,并了解了硬件描述语言 HCL 的使用。
    \item 通过在 PIPE 处理器中实现 \texttt{iaddq} 指令,理解了 PIPE 处理器流水线设计的特点和原理,并掌握了流水线处理器中的指令实现方法。
    \item 通过对 \texttt{ncopy.ys} 的优化,深入理解了流水线处理器中的性能瓶颈与优化策略,掌握了循环展开、指令调度、预加载等底层优化技术。
\end{itemize}
\end{document}